\documentclass{article}
\usepackage{graphicx} % Required for inserting images

\title{Writefull Testing}
\author{Brittany Channer}
\date{November 2023}

\begin{document}

\maketitle
\begin{abstract}
    UK garage, abbreviated as UKG, is a genre of electronic dance music which originated in England in the early to mid-1990s. The genre was most clearly inspired by garage house, but also incorporates elements from dance-pop, R\&B, and jungle. It is defined by percussive, shuffled rhythms with syncopated hi-hats, cymbals, and snares, and may include either 4/4 house kick patterns or more irregular "2-step" rhythms. Garage tracks also commonly feature 'chopped up' and time-stretched or pitch-shifted vocal samples complementing the underlying rhythmic structure at a tempo usually around 130 BPM. 

UK garage encompassed subgenres such as speed garage and 2-step, and was then largely subsumed into other styles of music and production in the mid-2000s, including bassline, grime, and dubstep. The decline of UK garage during the mid-2000s saw the birth of UK funky, which is closely related. AN EDIT WAS MADE.
\end{abstract}

\section{Introduction}

This exhaustive insider survey of UK garage music – whose demise has been greatly exaggerated – leaves you feeling like some wrung-out Ayia Napa afterparty casualty. It isn’t too great at clearly delineating the scene’s origins (in short: up-tempo versions of Paradise Garage-style house, hence the name, dovetailing with British jungle), and gives too much airtime to seminal club night Garage Nation, possibly because its former promoter Terry Stone co-directed this film. But, calling on an A-Z of exponents – from Pied Piper and MC Neat to approximately 13\% of So Solid Crew – it supplies a wry rewind to late 90s extravagance and hand-wringing consternation about the genre’s future.

The film meanders through UK garage’s early days, a “wild west” with promoters such as Stone running free in the era’s more laissez-faire clubland to road test MCs and dubplates of potential new bangers. The garage sound was more lacquered and luxuriating than jungle, bringing in more female clubbers, and the style 100\% aspirational. The documentary works best in the more structured segments cringing at the fashion of the era (think palm tree-print Moschino T-shirts and pinstripe trousers) and the excess. By the early 00s, when UK garage was decamping en masse for summer rec in shellshocked Cyprus, it was getting out of hand. Perennial afterparty venue Insomnia apparently had a secret room with a four-poster bed and, for some reason, a monkey; So Solid’s Lisa Maffia confides that the smell of sambuca hits her sick trigger to this day


Famously, Ayia Napa became quite violent, too, with So Solid’s MC Harvey and the rising star of UK garage’s successor grime Dizzee Rascal suffering knife attacks in successive years. It’s a mystery why garage was so blase about violence, especially when the music didn’t draw thematically from it as much as grime. (The film, flunking its sociology and history, doesn’t really follow this up.) A garage revival is now under way, although in a sanitised, nostalgic form that survives mostly at specialist festivals. As the pensive DJ Majestic points out, it’s cut off from the grassroots that has given grime and drum’n’bass longer legs. All the same, this is a messy salute to the evolutionary capacities and vitality of British electronic music.



















\end{document}
